\documentclass[a4paper,11pt]{article}

\usepackage[T1]{fontenc}
\usepackage[utf8]{inputenc}
\usepackage{graphicx}
\usepackage{xcolor}

\renewcommand\familydefault{\sfdefault}
\usepackage[defaultmono]{droidmono}

\usepackage{enumerate}
\usepackage{hyperref} 

\usepackage{geometry}
\geometry{total={210mm,297mm},
left=25mm,right=25mm,%
bindingoffset=0mm, top=20mm,bottom=20mm}


\linespread{1.3}

\newcommand{\linia}{\rule{\linewidth}{0.5pt}}

\makeatletter
\renewcommand{\maketitle}{
\begin{center}
\vspace{2ex}
{\huge \textsc{\@title}}
\vspace{1ex}
\\
\linia\\
\@author \hfill \@date
\vspace{4ex}
\end{center}
}
\makeatother
%%%
\usepackage{fancyhdr}
\pagestyle{fancy}
\lhead{}
\chead{}
\rhead{}
\lfoot{ChallP FS16}
\cfoot{}
\rfoot{Page \thepage}
\renewcommand{\headrulewidth}{0pt}
\renewcommand{\footrulewidth}{0pt}
%

% code listing settings
\usepackage{listings}
\lstset{
    language=Python,
    basicstyle=\ttfamily\small,
    aboveskip={1.0\baselineskip},
    belowskip={1.0\baselineskip},
    columns=fixed,
    extendedchars=true,
    breaklines=true,
    tabsize=4,
    prebreak=\raisebox{0ex}[0ex][0ex]{\ensuremath{\hookleftarrow}},
    frame=lines,
    showtabs=false,
    showspaces=false,
    showstringspaces=false,
    keywordstyle=\color[rgb]{0.627,0.126,0.941},
    commentstyle=\color[rgb]{0.133,0.545,0.133},
    stringstyle=\color[rgb]{01,0,0},
    numbers=left,
    numberstyle=\small,
    stepnumber=1,
    numbersep=10pt,
    captionpos=t,
    escapeinside={\%*}{*)}
}

%%%----------%%%----------%%%----------%%%----------%%%

\begin{document}

\title{Z-Wave Factsheet}

\author{fbinna, vmeier, laquino}

\date{2016}

\maketitle

\section*{Generell}
{\renewcommand{\arraystretch}{1.5}
	\begin{tabular}{| p{3.5cm} | p{10cm} |}
		\hline
		\textbf{Einsatzgebiet} & Home-Automation \\\hline
		\textbf{Topologie} & Mesh \\\hline
		\textbf{Range} & bis zu 100m zwischen 2 Nodes \\\hline
		\textbf{Maximaler Range} & Durchschnittlich 200m (Weiterleitung über 4 Hops) \\\hline
		\textbf{Anzahl Geräte pro Netzwerk} & 232 \\\hline
		
	\end{tabular}
}
\subsection*{Gerätetypen}
\begin{itemize}
 \item Elektrische Schalter (ergänzend oder Ersatz für physischen Schalter)
 \item Elektrische Dimmer (ergänzend oder Ersatz für physischen Schalter)
 \item Motoren (zum Öffnen / Schliessen von Fenstern, Rollläden etc.)
 \item Bildschirme, LED-Displays, Sirenen etc. (Signal-Quellen)
 \item Sensoren (Themometer, Hygrometer etc.)
 \item Thermostat Controller (Thermostat Radiator Valves, Bodenheizung-Controller etc.)
 \item Fernbedienungen (Universelle Infrarot-Fernbedienungen, spezielle Z-Wave Fernbedienungen ...)
 \item USB Sticks und IP Gateways (für Fernsteuerung über PC / über Internet)
\end{itemize}

\subsection*{Netzwerk-Informationen}
Ein Netzwerk besteht jeweils aus einem (oder in Spezialfällen aus mehreren) Controller und mehreren Slaves. Die Komponenten können über eine HomeID und NodeID eindeutig identifiziert werden.\\\\
{\renewcommand{\arraystretch}{1.5}
	\begin{tabular}{| p{3.5cm} | p{10cm} |}
		\hline
		\textbf{HomeID} & 32bit; Hersteller weist dem Controller eine ID zu, der die HomeID des Netzwerks bestimmt\\\hline
		\textbf{NodeID} & 8bit; Controller weist allen Slaves eine NodeID zu\\\hline
	\end{tabular}
}

\section*{Protokoll-Spezifikationen}

\subsection*{Physical Layer}
{\renewcommand{\arraystretch}{1.5}
	\begin{tabular}{| p{3.5cm} | p{10cm} |}
		\hline
		\textbf{Datenübertragungs-rate} & 9.6kbps / 40kbps / 100kbps\\\hline
		\textbf{Frequenz} &  {\textbf{Europa}: 868.42 MHz; \textbf{USA}: 908.42 MHz} \\\hline
		\textbf{Encoding} & Machester (9.6kbps) / NRZ (40kbps / 100kbps) \\\hline
		\textbf{Maximaler Range} & Durchschnittlich 200m (Weiterleitung über 4 Hops) \\\hline
		\textbf{Anzahl Geräte pro Netzwerk} & 232 \\\hline
	\end{tabular}
}

\subsection*{Security}
{\renewcommand{\arraystretch}{1.5}
	\begin{tabular}{| p{3.5cm} | p{10cm} |}
		\hline
		\textbf{Preshared Key} & 128bit Network Key; Von Controller generiert\\\hline
		\textbf{Cipher \& MAC-Keys} & 128bit; Von Netzwerk Key abgeleitet\\\hline
		\textbf{None} & 64bit; Gegen Replay-Attacken\\\hline
		\textbf{Encryption} & AES-OFB\\\hline
		\textbf{Data-Authentication} & AES-CBCMAC\\\hline
	\end{tabular}
}


\subsection*{Frame}
\begin{figure}[h!t]
	\includegraphics{images/frame.png}
\end{figure}

\section*{Quellen}

\href{http://www.z-wave.com/faq}{Offizielles Z-Wave FAQ}\\
\href{https://www.youtube.com/watch?v=KYaEQhvodc8}{BlackHat 2013 - Hacking Z-Wave}\\
\href{http://wiki.zwaveeurope.com/index.php?title=Z-Wave_Technical_Handbook}{Z-Wave Europe Wiki - Handbook (Linksammlung)}\\
\href{http://wiki.zwaveeurope.com/index.php?title=Z-Wave_Application_Layer}{Z-Wave Europe Wiki - Application Layer Details}


\end{document}
